\documentclass[a4paper,11pt]{article} 
\usepackage[polish]{babel} 
\usepackage[utf8]{inputenc}
\usepackage{times}
\usepackage{graphicx}
\usepackage{anysize}
\marginsize{3cm}{3cm}{3cm}{3cm}

\begin{document}

\section*{Całka Riemanna}
\label{sec:calka}

\flushleft{Niech dana będzie funkcja ograniczona $f\colon[a, b]\to R.$ \emph{Sumą częściową} (Riemanna) nazywa się liczbę 

$$R_{f, P(q_1, , q_n)} = \sum_{i = 1}^n f(q_i) \cdot \Delta p_i.$$ 

Funkcję f nazywa się całkowalną w sensie Riemanna lub krótko R-całkowalną, jeśli dla dowolnego ciągu normalnego $(P^k)$ podziałów przedziału $[a, b]$, istnieje (niezależna od wyboru punktów pośrednich) granica

$$R_f = \lim_{k \to \infty} R_{f, P^k q_1^k, , q_{n_k}^k}$$

nazywana wtedy całką Riemanna tej funkcji. Równoważnie: jeżeli istnieje taka liczba $R_f$, że dla dowolnej liczby rzeczywistej  $\varepsilon > 0$ istnieje taka liczba rzeczywista $\delta > 0$, że dla dowolnego podziału $P(q_1, \dots, q_n)$ o średnicy ${diam}\; P(q_1, \dots, q_n) < \delta;$ bądź też w języku rozdrobnień: że dla dowolnej liczby rzeczywistej $\varepsilon > 0$ istnieje taki podział $S(t_1, \dots, t_m)$ przedziału $[a, b]$, że dla każdego podziału $P(q_1, \dots, q_n)$ rozdrabniającego $S(t_1, \dots, t_m)$ zachodzi

$$R_{f, P(q_1, , q_n)} - R_f < \varepsilon.$$

Funkcję f nazywa się wtedy całkowalną w sensie Riemanna (R-całkowalną), a liczbę $R_f$ jej całką Riemanna.}

\begin{figure}[h]
\centerline{\includegraphics{calka-riemanna}}
\caption{Przykłady sum Riemanna}
\end{figure}

\end{document}
